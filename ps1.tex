% This LaTeX was auto-generated from MATLAB code.
% To make changes, update the MATLAB code and export to LaTeX again.

\documentclass{article}

\usepackage[utf8]{inputenc}
\usepackage[T1]{fontenc}
\usepackage{lmodern}
\usepackage{graphicx}
\usepackage{color}
\usepackage{listings}
\usepackage{hyperref}
\usepackage{amsmath}
\usepackage{amsfonts}
\usepackage{epstopdf}
\usepackage{matlab}

\sloppy
\epstopdfsetup{outdir=./}
\graphicspath{ {./ps1_images/} }

\begin{document}

\matlabtitle{Homework One}

\begin{par}
\begin{flushleft}
clc; close all; clear all; 
\end{flushleft}
\end{par}


\matlabheading{Problem 1.1 (Vectors)}

\begin{matlabcode}
x = [1i, 0, 1]; y = [1, -1, 2j]; 
% (a) Calculate the length of x and y
fprintf('The length of x is %01.2f and y is %01.2f', norm(x), norm(y));
\end{matlabcode}
\begin{matlaboutput}
The length of x is 1.41 and y is 2.45
\end{matlaboutput}
\begin{matlabcode}
% (b) Inner products
fprintf(['The inner products are conugate identical. \n'...
         '<x, y> = %01.1f + %01.1fj; <y, x> = %01.1f + %01.1fj.'], ...
         dot(x,y), imag(dot(x,y)), dot(x,y), imag(dot(y,x))); 
\end{matlabcode}
\begin{matlaboutput}
The inner products are conugate identical. 
<x, y> = 0.0 + 1.0j; <y, x> = 0.0 + -1.0j.
\end{matlaboutput}
\begin{matlabcode}
% (c) Give an example of a vector orthogonal to x and y
a = sym('a'); b = sym('b'); 
V = [a, b, 1]; 
[a, b] = solve([0; 0] == [dot(x,V); dot(y,V)], [a, b]); 
V = [a, b, 1]
\end{matlabcode}
\begin{matlabsymbolicoutput}
V = 
    $\displaystyle \left(\begin{array}{ccc}
-i & -3 i & 1
\end{array}\right)$
\end{matlabsymbolicoutput}
\begin{matlabcode}
[dot(x, V), dot(y,V)]
\end{matlabcode}
\begin{matlabsymbolicoutput}
ans = 
    $\displaystyle \left(\begin{array}{cc}
0 & 0
\end{array}\right)$
\end{matlabsymbolicoutput}


\matlabheading{Matrix Multiplication}

\begin{par}
\begin{flushleft}
(a) Write out by hand
\end{flushleft}
\end{par}

\vspace{1em}

\vspace{1em}

\vspace{1em}

\begin{par}
\begin{flushleft}
(b) Write out by hand
\end{flushleft}
\end{par}

\vspace{1em}

\vspace{1em}

\vspace{1em}

\begin{matlabcode}
% (c)
A  = [1, 1, 1; 1, -1, 1; -1, -1, 1]; 
B  = [1, 2, 1; 0,  3, 1;  0,  0, 1];
A*B
\end{matlabcode}
\begin{matlaboutput}
ans = 3x3    
     1     5     3
     1    -1     1
    -1    -5    -1

\end{matlaboutput}
\begin{matlabcode}
B*A
\end{matlabcode}
\begin{matlaboutput}
ans = 3x3    
     2    -2     4
     2    -4     4
    -1    -1     1

\end{matlaboutput}

\begin{par}
\begin{flushleft}
This tells us matrix multiplication is not commutitive. 
\end{flushleft}
\end{par}


\matlabheading{Matrix calculations}

\begin{matlabcode}
A = [1, 1, 1, 1; 1, -1, 1, -1; 1, 1, -1, -1; 1, -1, -1, 1];
trace(A); 
det(A) % By hand, break into successive minors until 2x2
\end{matlabcode}
\begin{matlaboutput}
ans = 16
\end{matlaboutput}
\begin{matlabcode}
eig(A) % As defined on the review sheet
\end{matlabcode}
\begin{matlaboutput}
ans = 4x1    
    -2
    -2
     2
     2

\end{matlaboutput}
\begin{matlabcode}
[V, ~] = eig(A)
\end{matlabcode}
\begin{matlaboutput}
V = 4x4    
         0   -0.5000    0.2887   -0.8165
   -0.7071    0.5000   -0.2887   -0.4082
    0.7071    0.5000   -0.2887   -0.4082
         0    0.5000    0.8660         0

\end{matlaboutput}

\end{document}
